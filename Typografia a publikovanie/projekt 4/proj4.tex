
\documentclass[a4paper, 11pt]{article}
\usepackage[text={17cm, 24cm}, left=2cm, top=3cm]{geometry}
\usepackage[utf8]{inputenc}
\usepackage[czech]{babel}
\usepackage{times}
\usepackage{hyperref}

\begin{document}
    \begin{titlepage}
        \begin{center}
            {\Huge \textsc{Vysoké učení technické v Brně}} \\ \medskip
            {\huge \textsc{Fakulta informačních technologií}} \\ 
            \vspace{\stretch{0.382}}
            {\LARGE Typografie a publikování\,--\,4. projekt} \\ \medskip
            {\Huge Bibliografické citace}
            \vspace{\stretch{0.618}}
        \end{center}
        {\Large \today \hfill Adrián Matušík}
    \end{titlepage}
Typografia sa zaoberá grafickou úpravou dokumentov s~použitím vhodných rezov písma a usporiadaním jednotlivých znakov a odsekov do~vhodnej, pre~čitateľa zrozumiteľnej a esteticky akceptovateľnej podoby. Zaoberá sa tiež dizajnom písma, výberom farebnej schémy dokumentov, ilustrácií, zalamovaním textu do~odsekov až po~výber papiera pre~tlač~\cite{wiki1}.

Kniha Typografie~\cite{bookTypography} je užitočným typografickým sprievodcom, ktorá by nemala chýbať vvknižnici žiadneho milovníka grafického designu. Dnes na~prácu s~typografiou existuje množstvo program, ja osobne používam \TeX. Pokiaľ ste v~obore typografie nováčikmi určite by som vám odporučil prečítať si knihu~\cite{bookBeginning}, ktorá vám poskytne dostatok informácií a znalostí na~vytvorenie profesionálne vyzerajúceho dokumentu. Pre~ľudí ktorí preferujú radšej študujú z~kníh v~elektronickej podobe je tu~\cite{onlineNavod}.

\TeX možno využiť aj pri~písaní správ výskumných projektov, pri~technickej dokumentacii, pri~písaní bakalárskych alebo diplomových prác. Alebo taktiež je možné spraviť vašu záverečnú prácu na~tému \LaTeX, jednou z~takýchto prácí je~\cite{zaverecnaPraca1}, v~ktorej sa autori venujú všeobecne všetkým možným informáciam a používaniu tohto programu.
Naopak konkrétnej téme a to transformácií výrazov z~editoru rovnic do~\LaTeX u sa venuje~\cite{zaverecnaPraca2}.

Na~internete ľahko nájdete množstvo tutoriálov a návodov ako v~\LaTeX u pracovať, môžu to byť články odborné, písané v~českom alebo slovenskom jazyku~\cite{onlineNavod2}~\cite{magazine}, alebo naopak písané vo~forme blogov, či časopisov a písané v~angličtine~\cite{journal1}~\cite{journal2}.
\bibliographystyle{plain}
\bibliography{proj4}
\end{document}
